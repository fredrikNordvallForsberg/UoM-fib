\documentclass{article}

\usepackage[margin=2cm]{geometry}
\usepackage{amsmath}
\usepackage{stmaryrd}
\usepackage[all]{xy}

\newtheorem{theorem}{Theorem}
\newtheorem{corollary}{Corollary}

\newcommand{\semRel}[1]{\llbracket #1 \rrbracket_{\mathsf{R}}}
\newcommand{\semGrp}[1]{\llbracket #1 \rrbracket_{\mathsf{G}}}
\newcommand{\sem}[1]{\llbracket #1 \rrbracket}

\newcommand{\Dom}{\mathsf{dom}}

\begin{document}

\paragraph{$G$-Set Model} Let $G$ be an abelian group. Using the
fibration $p : \mathrm{Mon}//\mathrm{Set} \to \mathrm{Mon}$, where the
objects of $\mathrm{Mon}//\mathrm{Set}$ are pairs $(M, (A, \psi))$ of
a monoid $M$ and a $M$-Set $(A, \psi)$. Use the notation
$|(M,(A,\psi))|$ to mean $A$, the underlying set of the semantic type
$(M,(A,\psi))$.

Interpretation of types, given some abelian group $G$:
\begin{displaymath}
  \begin{array}{l@{\hspace{0.3em}}c@{\hspace{0.3em}}l}
    \semGrp{\Delta \vdash \mathsf{quantity}(D)} &=&
    (G^{|\Delta|}, G, \psi(g,g_1) = (\sem{D}g)g_1) \\
    \semGrp{\Delta \vdash T \times U} &=&
    (G^{|\Delta|}, |\semGrp{T}| \times |\semGrp{U}|, \psi(g,(t,u)) = (\psi_T(g,t), \psi_U(g,u))) \\
    \semGrp{\Delta \vdash T \to U} &=&
    (G^{|\Delta|}, |\semGrp{T}| \to |\semGrp{U}|, \psi(g,f) = \lambda x.~\psi_U(g, f~(\psi_T(g^{-1}, x)))) \\
    \semGrp{\Delta \vdash \forall X.T} &=&
    (G^{|\Delta|}, \{ x \in |\semGrp{T}| \mathrel| \forall g \in G.~\phi_T((e,g),x) = x \}, \psi(g,x) = \psi_T((g,e_G),x))
  \end{array}
\end{displaymath}
Note that the monoid component on the RHS of each clause here is an
abelian group, so we are justified in using the inverse in the clause
for function types.

\paragraph{Relational Model} Each type is interpreted as a triple $(n, A, P)$,
where $n$ is a natural number, $A$ is a set, and $P \subseteq G^n
\times A \times A$. Use the notation $|(n,A,P)|$ to denote $A$, the
underlying set of this semantic type.

Relational semantics of types:
\begin{displaymath}
  \begin{array}{l@{\hspace{0.3em}}c@{\hspace{0.3em}}l}
    \semRel{\Delta \vdash \mathsf{quantity}(D)} & = & (|\Delta|, G, \{ (g, g_1, g_2) \mathrel| (\sem{D}g)g_1 = g_2 \}) \\
    \semRel{\Delta \vdash T \times U} & = &
    (|\Delta|,
    \begin{array}[t]{@{}l}
      |\semRel{T}| \times |\semRel{U}|, \\
      \{ (g, (t_1,u_1), (t_2,u_2)) \mathrel| (g,t_1,t_2) \in \semRel{T}, (g,u_1,u_2) \in \semRel{U} \}
    \end{array}
    \\
    \semRel{\Delta \vdash T \to U} & = &
    (|\Delta|,
    \begin{array}[t]{@{}l}
      |\semRel{T}| \to |\semRel{U}|, \\
      \{ (g, f_1, f_2) \mathrel| \forall t_1, t_2.~(g,t_1,t_2) \in \semRel{T} \implies (g,f_1t_1, f_2t_2) \in \semRel{U} \}
    \end{array}
    \\
    \semRel{\Delta \vdash \forall X.~T} & = & (|\Delta|,
    \begin{array}[t]{@{}l}
      \{ t \in |\semRel{T}| \mathrel| \forall g \in G.~((e_{G^{|\Delta|}},g), t, t) \in \semRel{T} \},\\
      \{ (g, t_1, t_2) \mathrel| \forall g' \in G.~((g,g'), t_1, t_2) \in \semRel{T} \} \}
    \end{array}
  \end{array}
\end{displaymath}

Note that this interpretation satisfies an analogue of identity
extension: for all $(n,A,P)$ that are the interpretation of types,
$(e,x_1,x_2) \in P$ iff $x_1=x_2$.

I think that this \textbf{isn't} the instantiation of the model in
Section 5.1 of the paper, with $Q_0 = G$ and $Q_r = \{ (g \in G, g_1,
g_2) \mathrel| g \cdot g_1 = g_2 \}$. So we might need to give a more
constrained model that the one constructed there.

\begin{theorem}
  For all types $\Delta \vdash T~\mathsf{Type}$, we have:
  \begin{enumerate}
  \item $|\semRel{T}| = |\semGrp{T}|$; and
  \item for all $g \in G^{|\Delta|}$ and $x_1, x_2 \in |\semRel{T}| =
    |\semGrp{T}|$, we have $(g, x_1, x_2) \in \semRel{T}$ iff
    $\psi_T(g,x_1) = x_2$.
  \end{enumerate}
  Note that, for all types $\Delta \vdash T$, the monoid component of
  the $G$-set interpretation is always $G^{|\Delta|}$, so the second
  statement is well-defined.
\end{theorem}

\noindent\emph{Proof.~} By induction on the derivation of $\Delta
\vdash T~\mathsf{Type}$:

\begin{itemize}
\item Case $T = \mathsf{quantity}(D)$. The underlying sets are both
  $G$, and we have $(g, g_1, g_2) \in \semRel{\mathsf{quantity}(D)}
  \Leftrightarrow (\sem{D}g)g_1 = g_2 \Leftrightarrow
  \psi_{\mathsf{quantity}(D)}(g,g_1) = g_2$, as required.
\item Case $T = U \times V$. By the induction hypothesis, we know that
  $|\semRel{U}| = |\semGrp{U}|$, and likewise for $V$, so the
  underlying sets of the two interpretations are equal. And we have,
  again using the induction hypothesis, $(g, (u_1,v_1), (u_2,v_2)) \in
  \semRel{U \times V} \Leftrightarrow (g, u_1, u_2) \in \semRel{U}
  \land (g,v_1,v_2) \in \semRel{V} \Leftrightarrow \psi_U(g,u_1) = u_2
  \land \psi_V(g,v_1) = v_2 \Leftrightarrow \psi_{U \times
    V}(g,(u_1,v_1)) = (u_2,v_2)$, as required.
\item Case $T = U \to V$. By the induction hypothesis, we know that
  $|\semRel{U}| = |\semGrp{U}|$ and likewise for $V$, so the
  underlying sets of the two intepretations are equal. And we have,
  again using the induction hypotheses, $(g, f_1, f_2) \in \semRel{U
    \to V} \Leftrightarrow (\forall u_1, u_2.~(g,u_1,u_2) \in
  \semRel{U} \implies (g,f_1u_1,f_2u_2) \in \semRel{V})
  \Leftrightarrow (\forall u_1, u_2.~\psi_U(g,u_1) = u_2 \implies
  \psi_V(g,f_1u_1) = f_2u_2) \Leftrightarrow (\forall u_1,
  u_2.~\psi_U(g^{-1},u_2) = u_1 \implies \psi_V(g,f_1u_1) = f_2u_2)
  \Leftrightarrow (\forall u_2.~\psi_V(g,f_1(\psi_U(g^{-1},u_2))) =
  f_2u_2) \Leftrightarrow \psi_{U \to V}(g,f_1) = f_2$, as required.
\item Case $T = \forall X.~U$. The two underlying sets are equal,
  using both points of the induction hypothesis for $U$:
  \begin{displaymath}
    \begin{array}{l@{\hspace{0.3em}}c@{\hspace{0.3em}}l}
      |\semRel{\forall X.U}|
      &=& \{ x \in |\semRel{U}| \mathrel| \forall g.~((e,g),x,x) \in \semRel{U} \} \\
      &=& \{ x \in |\semGrp{U}| \mathrel| \forall g.~((e,g),x,x) \in \semRel{U} \} \\
      &=& \{ x \in |\semGrp{U}| \mathrel| \forall g.~\psi_U((e,g),x) = x \} \\
      &=& |\semGrp{\forall X.U}|
    \end{array}
  \end{displaymath}
  The agreement of the relational intepretaton and the group action:
  \begin{displaymath}
    \begin{array}{l@{\hspace{0.3em}}c@{\hspace{0.3em}}l}
      (g,x_1,x_2) \in \semRel{T}
      &=& \forall g'.~((g,g'),x_1,x_2) \in \semRel{U} \\
      &=& \forall g'.~\psi_U((g,g'),x_1) = x_2 \\
      &=& \forall g'.~\psi_U((g,e),\psi_U((e,g'),x_1)) = x_2 \\
      &=& \psi_U((g,e),x_1) = x_2 \\
      &=& \psi_{\forall X.U}(g,x_1) = x_2
    \end{array}
  \end{displaymath}
  where we have used the ``stability'' property of inhabitants of
  $\forall$-types in the $G$-sets model.
\end{itemize}

\begin{corollary}
  The theorem has the following two consequences, which allow the
  transport of results between the two models:
  \begin{enumerate}
  \item For any pair of types $\Delta \vdash T$, $\Delta \vdash U$,
    they are isomorphic in the $G$-set model iff they are isomorphic
    in the relational model.
  \item For any type $\Delta \vdash T$, it is inhabited in the $G$-set
    model iff it is inhabited in the relational model.
  \end{enumerate}
\end{corollary}

\end{document}

\documentclass[a4paper,UKenglish]{lipics}
%\usepackage{etex}
%Packages
\usepackage{authblk}
\RequirePackage{xcolor}
\RequirePackage{stmaryrd}

\RequirePackage{amssymb,amsthm}
\RequirePackage{xspace}
\usepackage{microtype}
\usepackage{bussproofs}
\usepackage{mathtools} % \coloneqq for :=
\usepackage{xspace}
\usepackage{enumerate}
\usepackage{proof}
%\usepackage[bookmarks,bookmarksopen,bookmarksdepth=2]{hyperref}
\usepackage[capitalise]{cleveref}
\usepackage{diagrams}
%\RequirePackage[all]{xy}
%\usepackage{wrapfig}
\newtheorem{proposition}[theorem]{Proposition}

%Bibliography
\bibliographystyle{plain}

%%%%%%TEXT%%%%%%%%%%%
%Notes
\newcommand\note[1]{{ \bf \textcolor{red} {\vspace{2mm}\; \\ Note: #1\\}}}
%\renewcommand\note[1]{}
%Uncomment to hide notes

%To Reference Later
\newcommand\reference{{\large \bf \textcolor{magenta} {\\ REFERENCE ME \\}}}

%Link to Document
\newcommand{\link}{{\large \bf \textcolor{purple} {\\ LINK TO DOCUMENT \\}}}

%Highlight
\newcommand{\highlight}[1]{\colorbox{yellow}{#1}}

%SystemF
\newcommand{\SystemF}{System F\xspace}
\newcommand{\STLC}{Simply Typed $\lambda$-Calculus\xspace}

\newcommand{\LamOneFib}{$\lambda_1$-fibration\xspace}
\newcommand{\LamOneFibs}{$\lambda_1$-fibrations\xspace}
\newcommand{\LamTwoFib}{$\lambda_2$-fibration\xspace}
\newcommand{\LamTwoFibs}{$\lambda_2$-fibrations\xspace}

\newcommand{\ChangeOfBase}{change-of-base\xspace}
\newcommand{\UoM}{Units of Measure\xspace}


%%%%%%%%MATHS%%%%%

%%%%%%Arrows%%%%%
%2 Cells
\newcommand{\lift}[2]{%
\setlength{\unitlength}{1pt}
\begin{picture}(0,0)(0,0)
\put(0,{#1}){\makebox(0,0)[b]{${#2}$}}
\end{picture}
}
\newcommand{\lowerarrow}[1]{%
\setlength{\unitlength}{0.03\DiagramCellWidth}
\begin{picture}(0,0)(0,0)
\qbezier(-28,-4)(0,-18)(28,-4)
\put(0,-14){\makebox(0,0)[t]{$\scriptstyle {#1}$}}
\put(28.6,-3.7){\vector(2,1){0}}
\end{picture}
}
\newcommand{\upperarrow}[1]{%
\setlength{\unitlength}{0.03\DiagramCellWidth}
\begin{picture}(0,0)(0,0)
\qbezier(-28,11)(0,25)(28,11)
\put(0,21){\makebox(0,0)[b]{$\scriptstyle {#1}$}}
\put(28.6,10.7){\vector(2,-1){0}}
\end{picture}
}

%Cartesian Morphisms
\newcommand{\cart}[1]{\ensuremath{#1^{\S}}}
\newcommand{\opcart}[1]{\ensuremath{#1_{\S}}}

%Other Arrows
\newcommand{\ra}{\rightarrow}
\newcommand{\la}{\leftarrow}
\newcommand{\inc}{\hookrightarrow}
%\newarrow {Dashto} {}{dash}{}{dash}>
%\newarrow {Embed} {>}{-}{-}{-}{>}
\newcommand{\eqra}{\rightarrow {\hspace{-4mm} \raisebox{-1.5mm}{$\scriptstyle \msf{eq}$}} \hspace{1.5mm}} %Equality preserving right arrow

\newcommand{\expra}{\Rightarrow}
\newcommand{\morphra}{\rightarrow}
\newcommand{\synra}{\rightarrow}


%Categories
\newcommand{\msf}[1]{\mathsf{#1}} %Maths Font
\newcommand{\Grp}{\msf{Grp}}
\newcommand{\Ab}{\msf{Ab}}
\newcommand{\Set}{\msf{Set}}
\newcommand{\Cat}{\msf{Cat}}
\newcommand{\Gpd}{\msf{Gpd}}
\newcommand{\Fib}{\msf{Fib}}
\newcommand{\Fam}[1]{\msf{Fam}(#1)}
\newcommand{\Sub}[1]{\msf{Sub}(#1)}
\newcommand{\LAb}{\msf{L}_{\msf{Ab}}}
\newcommand{\Nom}{\msf{Nom}}
\newcommand{\terminal}{\msf{1}}

\newcommand{\ClLD}{\mathcal{C\ell}(\lambda{}D)} %Classifying category

\newcommand{\Rel}{\msf{Rel}}
\newcommand{\RelE}{\Rel(\E)}
\newcommand{\RelNom}{\Rel(\Nom)}

\newcommand{\A}{\mathcal{A}}
\newcommand{\B}{\mathcal{B}}
\newcommand{\C}{\mathcal{C}}
\newcommand{\D}{\mathcal{D}}
\newcommand{\E}{\mathcal{E}}


\newcommand{\fibre}[2]{#1_{_{#2}}}
\newcommand{\fibreE}[1]{\E_{#1}}
\newcommand{\fibreRelE}[1]{\RelE_{#1}}


\newcommand{\blank}{\, \underline{\hspace{2mm}} \,}

%Functors
\newcommand{\SqFun}[1][\blank]{[\, #1 \, ]} %[_]
\newcommand{\BracFun}[1][\blank]{(\, #1 \, )}
\newcommand{\EqFun}{\msf{Eq} \; }



%Groups
\newcommand{\PermA}{\msf{Perm} \; \mathbb{A}}

%GSets
\newcommand{\act}[1]{\cdot_{_{#1}}}
\newcommand{\SetAct}[1]{(#1, \; \cdot_{_{#1}})}
\newcommand{\GroupSet}[1]{#1/\!/\Set}
\newcommand{\GrpSet}{\GroupSet{\Grp}}
\newcommand{\CatSet}{\GroupSet{\Cat}}
\newcommand{\GpdSet}{\GroupSet{\Gpd}}
\newcommand{\sslice}{/\!/}
\newcommand{\AbSet}{\GroupSet{\Ab}}
\newcommand{\GSet}{\GroupSet{G}}
\newcommand{\PermASet}{\GroupSet{\PermA}}

%BB Sets
\newcommand{\bbA}{\mathbb{A}}
\newcommand{\bbN}{\mathbb{N}}
\newcommand{\bbQ}{\mathbb{Q}}
\newcommand{\bbZ}{\mathbb{Z}}

%HomSet
\newcommand{\Hom}[1][]{\mathrm{Hom}_{#1}}

%Types
\newcommand{\type}[1][T]{\Gamma \vdash #1 \; \msf{ Type}}
\newcommand{\term}[1][t:T]{\Gamma ; \Delta \vdash #1}
\newcommand{\ntup}[1]{\mkern 1.5mu\overline{\mkern-2.5mu#1\mkern-1.5mu}\mkern 1.5mu} %n-tuple


%%%%%%Brackets%%%%%
%Regular Brackets
\newcommand{\bracket}[1]{\left( #1 \right)}
\newcommand{\bbracket}[1]{\bigl( #1 \bigr)}
\newcommand{\Bbracket}[1]{\Bigl( #1 \Bigr)}
\newcommand{\grph}[1]{\ensuremath{\langle #1 \rangle}}


%Semantics
\newcommand{\sem}[1]{\ensuremath{\llbracket #1 \rrbracket} \;}
\newcommand{\semo}[1]{\ensuremath{\llbracket #1 \rrbracket _o} \;}
\newcommand{\semr}[1]{\ensuremath{\llbracket #1 \rrbracket _r} \;}

%%%Units of Measure%%%%
\newcommand{\unitTy}{\msf{1}}
\newcommand{\emptyTy}{\msf{0}}
\newcommand{\qnt}{\msf{quantity}}
\newcommand{\sreal}{\msf{real}}
\newcommand{\bool}{\msf{bool}}
\newcommand{\lengthDim}{\mathit{Length}}
\newcommand{\timeDim}{\mathit{Time}}
\newcommand{\massDim}{\mathit{Mass}}
\newcommand{\UoMterm}{ \Delta ; \Gamma \vdash t :T }
\newcommand{\UoMtype}{\Delta \vdash T \; \msf{ Type}}
\newcommand{\UoMFibration}{UoM-fibration\xspace}
\newcommand{\UoMFibrations}{UoM-fibrations\xspace}

\newcommand{\Deltadim}{\ensuremath{\Delta_{dim}}}
\newcommand{\Gammaops}{\ensuremath{\Gamma_{ops}}}
\newcommand{\etaops}{\ensuremath{\eta_{ops}}}


%%%Judgements%%%
\newcommand{\cj}[2]{#1 \vdash #2 \; \msf{ Ctx}}
\newcommand{\Tj}[2]{#1 \vdash #2 \; \msf{ Type}}
\newcommand{\Dj}[2]{#1 \vdash #2 \; \msf{ Dim}}
\newcommand{\tj}[4]{#1;#2 \vdash #3 : #4}

\newcommand{\proj}{\mathsf{pr}}
\newcommand{\inj}{\mathsf{inj}}
\newcommand{\emptyelim}{\mathsf{case}\,}
\newcommand{\inl}{\mathsf{inl}}
\newcommand{\inr}{\mathsf{inr}}
\newcommand{\case}[5]{\mathsf{case}\,#1\,\mathsf{of}\,\{\inj_1\,#2 \mapsto\,#3; \inj_2\,#4\mapsto\,#5\}}
\newcommand{\pack}[2]{\mathsf{pack}(#1,#2)}
\newcommand{\unpack}[4]{\mathsf{unpack}\,#1\,\mathsf{as}\,(#2,#3)\,\mathsf{in}\,#4}
\newcommand{\Dim}{D}
\newcommand{\Dvar}{X}

\newcommand{\id}{\mathrm{id}}

% Author macros %%%%%%%%%%%%%%%%%%%%%%%%%%%%%%%%%%%%%%%%
\title{Models for polymorphism over physical dimensions}
\author[1]{Robert Atkey}
\author[2]{Neil Ghani}
\author[2]{Fredrik Nordvall Forsberg}
\author[2]{Timothy Revell}
\author[3]{Sam Staton}
\affil[1]{University of Edinburgh}
\affil[2]{University of Strathclyde}
\affil[3]{University of Cambridge}
\authorrunning{Atkey, Ghani, Nordvall Forsberg, Revell, and Staton}
\Copyright{Robert Atkey and Neil Ghani and Fredrik Nordvall Forsberg and Timothy Revell and Sam Staton}%mandatory, please use full first names. LIPIcs license is "CC-BY";  http://creativecommons.org/licenses/by/3.0/

\subjclass{F.3.2 Semantics of Programming Languages.}% mandatory: Please choose ACM 1998 classifications from http://www.acm.org/about/class/ccs98-html . E.g., cite as "F.1.1 Models of Computation".
\keywords{Category Theory, Units of Measure, Dimension Types, Type Theory.}% mandatory: Please provide 1-5 keywords
%%%%%%%%%%%%%%%%%%%%%%%%%%%%%%%%%%%%%%%%%%%%%%%%%%%%%%%%

\begin{document}

\maketitle

\begin{abstract}
We provide a categorical framework for models of a type theory that has special types for physical quantities. The types are indexed by the physical dimensions that they involve. We use fibrations to organize this index structure in the models of the type theory. We develop some informative models of this type theory: firstly, a model based on group actions, which captures invariance under scaling, and secondly, a way of constructing new models using relational parametricity.

%  In 1998, Andrew Kennedy wrote a paper entitled Relational Parametricity and \UoM\cite{Kennedy:1997:RPU:263699.263761}. Inspired by the benefits afforded by the correspondence between $\lambda$-calculus and category theory, in this paper we give a categorical semantics to complement Kennedy's \UoM. We do this so as to make Kennedy's work more accessible to those of a semantic disposition and to discover natural extensions and alternative proofs of Kennedy's results suggested by the categorical perspective.
%
%  We start by following the standard approach in categorical logic by using fibrations to separate index-information (the units) from indexed-information (the types which may contain units). This leads us to the notion of a UoM-fibration. We prove some basic theorems about UoM-fibrations and give a variety of examples of \UoMFibrations, including a novel model based upon $G$-sets. We then explore parametricity for \UoM.
\end{abstract}

\section{Introduction}
This paper is about semantic models of programs that manipulate physical quantities, such as length or time. To measure a physical quantity we use units, such as metres, inches, seconds or hours. The particular units that we choose do not effect how we can describe a physical system, or the programs that we can write about it, and so in this paper our focus is on \emph{dimensions}.

A dimension is simply a physical quantity that can be measured. A fundamental principle of dimensions is that it is not meaningful to add or compare quantities of different dimension, but they can be multiplied.

Here is a simple polymorphic program that is defined for all dimensions; it takes a quantity $x$ of a given dimension $X$, and returns its double.
\begin{equation}
f\coloneqq (\Lambda X.\,\lambda x:\qnt(X).\,x+x)
:\ \forall X.\,\qnt(X)\to \qnt(X)
\label{eqn:double}
\end{equation}
To illustrate, we can use the polymorphic function $f$ to double a length of $5$ metres.
\begin{equation}
f_\lengthDim\,(5\mathrm{m})=
10\mathrm{m}:\qnt(\lengthDim)
\label{eqn:doubleapp}
\end{equation}

\noindent There are a few key points that are worth emphasising about examples \eqref{eqn:double} and \eqref{eqn:doubleapp} above:
\begin{itemize}
\item There are two kinds of variable, $\Dvar$ and $x$. The first variable $\Dvar$ stands for a dimension whereas $x$ stands for an inhabitant of a type. To emphasise this distinction, we use different abstraction symbols ($\lambda$ and $\Lambda$) for the two kinds of variable.
\item The type $\qnt(\Dvar)$ depends on a dimension $\Dvar$, and it is inhabited by quantities of that dimension. For example, the standard unit of measurement for length, the metre, is a quantity of that dimension, i.e. a constant $\mathrm m:\qnt(\lengthDim)$.
\end{itemize}

Several authors have developed programming languages with type systems that support physical quantities. Our starting point is the work of Kennedy~\cite{Kennedy:1997:RPU:263699.263761} who developed techniques for reasoning about these kinds of programs. In this we take a different approach by developing a general categorical notion of model for a programming language of this form, and by developing ways of building models.

\note{Add references.}



The main contributions of this paper are as follows.
\begin{enumerate}
\item We provide a general notion of a model for a programming language with dimension types by introducing the concept of a \emph{$\lambda D$-model} (Section~\ref{sec:sem}). The basic idea is that, for each context of dimension variables, there is a model of the simply typed $\lambda$-calculus extended with types of quantities of the dimensions definable in the context ($\qnt(D)$ etc.). Moreover these models of the simply typed $\lambda$-calculus are related by substituting for dimension variables, and this also defines a universal property for polymorphic quantification over dimension variables.

\item An important example of a $\lambda D$-model is the actions of scaling groups (Section~\ref{sec:MonSet}). This is as close as we can get to a set-theoretic semantics for programs with parametric dimension polymorphism. A difficulty here is, how does one understand $\qnt(\Dvar)$ as a set, if the dimension $\Dvar$ is not specified and we have no fixed units of measure for $\Dvar$? We resolve this by interpreting $\qnt(\Dvar)$ as the set of magnitudes, i.e.~positive real numbers, thought of as quantities of some unspecified unit of measure, but then by equipping $\qnt(\Dvar)$ with an action of the scaling group, to explain how to change the units of measure. We can then ask that any function $\qnt(\Dvar)\to\qnt(\Dvar)$ is invariant under changing that unspecified unit of measure, more precisely, invariant under scaling.

\note{Do we actually do this point? And does it really describe a contribution?}

\item We give new ways of building models, by extending relational parametricity techniques for typed lambda calculus to the setting with dimensions (Section~\ref{sec:param}).
\end{enumerate}


\section{Types with Physical Dimensions}
\label{sec:Not}
We begin by recalling a simple type theory, which we call $\lambda D$, indexed by dimensions, based on Kennedy's work~\cite{Kennedy:1997:RPU:263699.263761}. Within this type theory we can express programs such as~(\ref{eqn:double}) and (\ref{eqn:doubleapp}). Since there are two kinds of variable, we have two kinds of context.

\vspace{3mm} \noindent {\bf Dimensions and Dimension Contexts:}
A dimension context $\Delta$ is a finite list of distinct
dimension variables.
A dimension-expression-in-context $\Dj\Delta \Dim$ is a monomial
$\Dim$ in the variables $\Delta$.
More precisely,
if $\Delta=(\Dvar_1\dots \Dvar_n)$ and $k_1\dots k_n\in\mathbb Z$
then $\Dj \Delta{\Dvar_1^{k_1}\dots \Dvar_n^{k_n}}$.
We can make the set $\{\Dim~|~\Dj \Delta \Dim\}$ an Abelian group under addition of
exponents, and indeed this is the free Abelian group on $\Delta$.
This universal property gives a notion of substitution on dimension expressions.
For example,
$\Dj{X,Z}{(X^2Y^3)[^{(XZ^2)}\!/\!_Y]=X^3Z^6}$.

%The Lawvere theory of Abelian groups $\LAb$ is the opposite of the Kleisli category of $\Ab$, with natural numbers as objects. We write $\msf{Ab}(\C)$ for the internal Abelian groups of a category $\C$, which are just the finite product-preserving functors from $\LAb$ to $\C$.\\

\vspace{3mm} \noindent {\bf Types:} Well-formed types are given by judgements of the form $\Tj\Delta  T$ where
$\Delta$ is a dimension context and these judgements are generated by the following rules.
\[\begin{array}{c}
\AxiomC{$\Dj\Delta \Dim$}
		\UnaryInfC{$\Tj\Delta {\qnt(\Dim)}$}
		\DisplayProof

\hspace{5mm}
\AxiomC{$\Tj {\Delta, \Dvar}  T$}
	\UnaryInfC{$\Tj \Delta{ \forall \Dvar.T} $}
	\DisplayProof
\hspace{5mm}
\AxiomC{$\Delta \vdash T \; \msf{ Type} \; \; \; \Delta \vdash U \; \msf{ Type}$}
	\UnaryInfC{$\Delta \vdash T \rightarrow U  \; \msf{ Type}$}
	\DisplayProof
\\ \\
\AxiomC{  $\phantom{\vdash T \; \msf{ Type}}$ }
		\UnaryInfC{$\Delta \vdash \unitTy \; \msf{ Type}$}
	\DisplayProof
\hspace{5mm}
\AxiomC{$\Delta \vdash T \; \msf{ Type} \; \; \; \Delta \vdash U \; \msf{ Type}$}
	\UnaryInfC{$\Delta \vdash T\times U \; \msf{ Type}$}
	\DisplayProof
\hspace{5mm}
\AxiomC{  $\phantom{ \vdash T \; \msf{ Type}}$ }
		\UnaryInfC{$\Delta \vdash \emptyTy \; \msf{ Type}$}
	\DisplayProof
\hspace{5mm}
\AxiomC{$\Delta \vdash T \; \msf{ Type} \; \; \; \Delta \vdash U \; \msf{ Type}$}
	\UnaryInfC{$\Delta \vdash T + U \; \msf{ Type}$}
	\DisplayProof
\\ \\
\end{array}\]
Notice that we do not have System-F-style polymorphism, but instead dimension polymorphism. Types can be parameterised by dimensions, but they cannot be parameterised by types, since we do no have type variables. From the Curry-Howard perspective this is a first-order-logic where the domain of discourse is the theory of Abelian groups and where there is a single atomic predicate, $\qnt$.

\vspace{3mm} \noindent {\bf Terms and Typing Contexts:} Well-formed typing contexts are given by judgements
$\cj\Delta\Gamma $ where $\Delta$ is a dimension context, $\Gamma$ is
of the form ${x_1 : T_1, \ldots, x_n:T_n}$ and there is a well-formed
typing judgement $\Tj\Delta{ T_i}$ for every $i$. Well-formed terms
are given by judgements $\tj \Delta \Gamma t T$ where there is a
well-formed typing context $\cj \Delta \Gamma$ and
a well-formed type $\Tj \Delta T $. The rules for
the type formers $\unitTy$, $\blank\times\blank$, $\blank+\blank$ and $\blank \! \ra \! \blank$ are the usual ones from
simply typed $\lambda$-calculus.

\[\begin{array}{c}%
\AxiomC{%
$\cj \Delta{\Gamma,\Gamma'}\quad
\Tj\Delta T$}
\UnaryInfC
{$\tj \Delta{\Gamma,x:T,\Gamma'}xT$}
\DisplayProof
\quad
\AxiomC{%
$
\tj\Delta {\Gamma,x:T} {t}{U}
$%
}
\UnaryInfC
{$\tj\Delta {\Gamma} {\lambda x.t}{T\to U}$}
\DisplayProof
\quad
\AxiomC{%
$
\tj\Delta {\Gamma} {t}{T\to U}\quad
\tj\Delta {\Gamma} {u}{T}
$%
}
\UnaryInfC
{$\tj \Delta{\Gamma}{t\,u}U$}
\DisplayProof
\\[5mm]
\AxiomC{%
$\cj\Delta {\Gamma}
$%
}
\UnaryInfC
{$\tj \Delta{\Gamma}{()}\unitTy$}
\DisplayProof
\qquad
\AxiomC{$\tj\Delta {\Gamma} {t_1}{T_1}$}
\AxiomC{$\tj\Delta {\Gamma} {t_2}{T_2}$}
\BinaryInfC
{$\tj \Delta{\Gamma}{(t_1,t_2)}{T_1\times T_2}$}
\DisplayProof
\qquad
\AxiomC{%
$
\tj\Delta {\Gamma} {t}{T_1\times T_2}
$%
}
\UnaryInfC
{$\tj\Delta {\Gamma} {\proj_i(t)}{T_i}$}
\DisplayProof
\\[5mm]
\AxiomC{%
$
\tj\Delta {\Gamma} {t}{\emptyTy}
\quad \Tj\Delta T$%
}
\UnaryInfC
{$\tj\Delta {\Gamma} {\emptyelim t}{T}$}
\DisplayProof
\qquad
\AxiomC{%
$
\tj\Delta {\Gamma} {t}{T_i}
\ (i\in\{1,2\})$}
\UnaryInfC
{$\tj\Delta {\Gamma} {\inj_i\,t}{T_1 + T_2}$}
\DisplayProof
\\[5mm]
\AxiomC{$\tj\Delta {\Gamma} {t}{T_1 + T_2}
\quad \big(\tj\Delta {\Gamma, x_i : T_i} {u_i}{U}\big)_{i\in\{1,2\}}$}
\UnaryInfC
{$\tj\Delta {\Gamma} {\case{t}{x_1}{u_1}{x_2}{u_2}}{U}$}
\DisplayProof
\end{array}\]
%
In addition, we have the introduction and elimination rules for quantification over a unit variable.
\[\begin{array}{cc}
\hspace{5mm}
\AxiomC{$\tj{\Delta,\Dvar}\Gamma tT$ }
		\UnaryInfC{$\tj \Delta  \Gamma {\Lambda \Dvar.t} {\forall u.T}$}
		\DisplayProof

&\hspace{5mm}
\AxiomC{$\Dj\Delta \Dim \quad \tj \Delta \Gamma  t{\forall \Dvar.T}$}
	\UnaryInfC{$\tj \Delta  \Gamma {t_\Dim}{T[\Dim/\Dvar]}$}
	\DisplayProof
%	\\ \\
%\hspace{5mm}
%\AxiomC{$\Delta\vdash \Dim \quad \tj {\Delta} \Gamma  t {T[\Dim/\Dvar]}$}
%	\UnaryInfC{$\tj \Delta  \Gamma {\pack e t}{\exists \Dvar.T}$}
%	\DisplayProof
%
%&\hspace{5mm}
%\AxiomC{$\tj {\Delta} \Gamma  t {\exists \Dvar.T}
%  \quad
%  \tj{\Delta,\Dvar}{\Gamma, x:T} u U
%  $}%
%	\UnaryInfC{$\tj \Delta  \Gamma {\unpack t \Dvar x u}{U}$}
%	\DisplayProof
%	\\ \\
\end{array}\]
%
We use $\bool$ as an abbreviation for $\unitTy + \unitTy$.
We may work with some dimension constants and term constants
by judging terms in a context $(\Deltadim;\Gammaops)$.
For instance, we could consider $\Deltadim = (\lengthDim, \timeDim)$ and
\[\begin{array}{r@{}l}\Gammaops = (
&\mathrm{m}:\qnt(\lengthDim),
\ \mathrm{s}:\qnt(\timeDim),
\\&+:\forall \Dvar. \qnt(\Dvar)\times  \qnt(\Dvar) \rightarrow \qnt(\Dvar),
\\&\times :\forall \Dvar_1. \forall \Dvar_2. \qnt(\Dvar_1)\times \qnt(\Dvar_2) \rightarrow \qnt(\Dvar_1 \cdot \Dvar_2),
\ 1:\qnt(1),
\\&\mathsf{inv}:\forall \Dvar.\qnt(\Dvar)\to \qnt (\Dvar^{-1}),
\ {<}:\forall \Dvar.\qnt(\Dvar)\times \qnt(\Dvar)\to \bool)\text.
\end{array}\]

One could also define a type of signed/zero quantities $\sreal(\Dvar)\coloneqq \qnt(\Dvar)+\unitTy+\qnt(\Dvar)$, and then extend the language with further arithmetic term constants such as signed addition $+:\forall \Dvar.\sreal(\Dvar)\times \sreal(\Dvar)\to\sreal(\Dvar)$.


This is an idealized language, designed to demonstrate polymorphism over dimension types. As such it is missing many important features for a general purpose language, such as recursive types and terms.

\note{Regarding recursion: I think we should leave it at this.
Otherwise, we have to start talking about algebraically compact fibrations, etc. WAS: In this paper we do not treat recursion, but a future extension of our work may look in the direction of Pitts' PolyPCF \cite{pitts2000parametric}.}





\section{Categorical Semantics of Dimension Types} \label{sec:sem}
Next up we give a general categorical semantics for the $\lambda D$ type theory. Central to this is the notion of a $\lambda \forall$-fibration.

\begin{definition}
A \emph{$\lambda\forall$-fibration} is a bicartesian closed fibration with simple products.
\end{definition}

It is well-known that $\lambda \forall$-fibrations give a categorical model of the fragment of first-order logic without existential quantifiers. Although this is well-known, we briefly introduce the basic notions now, since they are central to our development. We refer to \cite{jacobs1999categorical} for the full details.

A fibration $p:\E\to\B$ is a functor between categories satisfying certain conditions. These conditions (along with the structure in Definition~\ref{def:lamDModel}) allow us to model the $\lambda D$ type theory. The basic idea is that dimension contexts will be interpreted as objects $\B$. Then for each $B\in\B$ we consider the full subcategory $\E_B$ of $\E$, with objects $E\in\E$ for which $p(E)=B$. The objects of $\E_B$ will be used to interpret types in dimension context $B$, and the morphisms in $\E_B$ will be used to interpret terms. We can substitute dimension expressions for dimension variables, and this substitution will be interpreted using morphisms in $\B$. Since $p$ is a fibration, for each morphism $f:B\to B'$ in $\B$ there is a canonical associated reindexing functor $f^\ast:\E_{B'}\to \E_B$, which we will use to describe substitution for dimension variables in types and terms.

A fibration is said to be bicartesian closed if $\E_B$ is a Cartesian closed category with coproducts for all $B$, and each reindexing functor $f^*:\E_{B'}\to\E_B$ preserves products, exponentials and coproducts. This bicartesian closed structure is needed to interpret the product, function and coproduct types.

Concatenation of dimension contexts will be interpreted using products in the category~$\B$. The reindexing functors  $\pi^\ast:\E_{B}\to \E_{B\times B'}$ for the product projections $\pi:B\times B'\to B$ correspond to context-weakening. A fibration $p:\E\to \B$ is said to have simple products if $\B$ has products and the reindexing functors for the product projections have right adjoints $\forall:E_{B\times B'}\to E_B$ that are compatible with reindexing (`Beck-Chevalley'). A fibration is said to have products if this condition holds for all morphisms in the base, not just projections. These right adjoints are needed to interpret universal quantification of dimension variables in types.

\begin{definition}\label{def:lamDModel}
A $\lambda D$-model $(p,G, Q)$ is a $\lambda \forall$-fibration $p:\E\to \B$, an Abelian group object $G$ in $\B$, and an object $Q$ in the fibre $\fibreE{G}$.
\end{definition}

Recall that an Abelian group object in a category $\B$ with products is given by an object $G$ together with maps $e:1\to G$, $m:G\times G\to G$ and $i:G\to G$ satisfying the laws of Abelian groups. This group structure is needed to interpret dimension expressions: for each vector of $n$ integers we have a morphism $G^n\to G$.

An equivalent way to define Abelian group objects if $\B$ has chosen products is as follows. Recall that the Lawvere theory for Abelian groups is the category $\LAb$ whose objects are natural numbers, and where a morphism $m\to n$ is an $m\times n$ matrix of integers. Composition of morphisms is given by matrix multiplication, and categorical products are given by arithmetic addition of natural numbers. An Abelian group object in $\B$ is an object $G$ of $\B$ together with a strictly-product-preserving functor $F:\LAb\to\B$ such that $F(1)=G$.

We remark that the Abelian group $G$ in a $\lambda D$-model is analogous to the generic object in a model of \SystemF.

In order to ascertain the value of Definition~\ref{def:lamDModel}, we need to do three things: i) show that a $\lambda D$-model does in fact provide categorical models of dimension types, ii) give some examples of $\lambda D$-models, and iii) prove some theorems to show the viability of reasoning at this level of abstraction.





\subsection{Modelling Dimension Types}
To show that $\lambda D$-models provide a categorical semantics for dimension types, we must show how to interpret the syntax given in Section~\ref{sec:Not} in any given $\lambda D$-model. We will use the $\lambda \forall$-fibration to separate the indexing information (the dimensions) from the indexed information (the types and terms). This means, that the base category of the fibration will be used to interpret dimension contexts, and types and terms will be interpreted as objects in the fibres above the dimension contexts in which they are defined. Cartesian closure of the fibres will allow us to inductively interpret types built from $\unitTy$, $\times$ and $\rightarrow$, and we will take the standard approach in categorical logic to interpret quantification of dimensions - by using right adjoints. Finally, since dimension expressions for a dimension context, are defined as elements of the free Abelian group on that dimension context, we will use the Abelian group object structure to interpret
such expressions. Formally, we interpret the syntax as follows.

\begin{itemize}
\item Dimension contexts $\Delta = X_1, \ldots, X_n$ are interpreted as the product of the Abelian group object $\sem{\Delta} = G^n$ in $\B$.
\item Dimension expressions $\Dj \Delta D$ are interpreted as morphisms $G^n \ra G$ in the base $\B$, by using the structure of the Abelian group object $G$. For example, $\sem{X_1, X_2 \vdash X_1 \cdot X_2^{-1}}(g_1,g_2) = g_1 \cdot g_2^{-1}$. We assume that there is an interpretation $d_i : 1 \to G$ for every primitive dimension constant $d_i \in \Deltadim$.

\item Well-formed types $\Delta \vdash T \; \msf{ Type}$ are interpreted as objects $\sem{T}$ in the fibre above $\sem{\Delta}$,  defined by induction on the structure of $T$. We interpret $\unitTy$, $\times$ and $\rightarrow$ using the Cartesian closed structure of the fibres, and quantification of a dimension variable $\sem{\Delta \vdash \forall X. T}$ is defined by right adjoint to reindexing along the projection $\pi: \sem{\Delta \vdash \Gamma, X} \rightarrow \sem{\Delta \vdash \Gamma}$. Quantities $\Delta \vdash \qnt(D)$ are interpreted by reindexing the object $Q$ along $\sem{\Dj \Delta D}$, i.e. $\sem{\Delta \vdash \qnt(D)} = \sem{\Dj \Delta D}^\ast (Q)$.

\item Well-formed typing contexts $\Delta \vdash \Gamma \; \msf{ ctxt}$ are interpreted as products in the fibre above $\sem{\Delta}$, i.e.\ $\sem{\Delta \vdash x_1 : T_1 ,..., x_n:T_n} = \sem{\Delta \vdash T_1} \times ... \times \sem{\Delta \vdash T_n}$.

\item Well-formed terms $\Delta, \Gamma \vdash t : T$ are interpreted as morphisms $\sem{t} : \sem{\Gamma} \ra \sem{T}$ in the fibre above $\sem{\Delta}$. We assume that there is an interpretation $\etaops : 1 \to \sem{\Gammaops}$ of all the primitive operations.
\end{itemize}

In this paper we have only considered universal quantification of units but existential quantification can be given just as easily. Existential quantification is interpreted as the left adjoint to reindexing along a projection. Properties of  existential quantification can be proven by dualising the relevant proofs of properties about universal quantification.


\subsection{First Examples of $\lambda D$-Models}
We now give some examples of $\lambda D$-models. We begin by noting that in Kennedy's paper, a simpler approach is taken to the semantics of dimensions, the dimensions/units are simply thrown away in a {\em unit-erasure semantics}. From the categorical perspective, this means the calculus is stripped of its fibred structure leaving only a simply typed $\lambda$-calculus, which Kennedy models, as to be expected, within a CCC. In particular, he chooses the CCC of complete partial orders which he needs for recursion. Nevertheless, Kennedy's model can be viewed as a $\lambda D$-model.


\begin{example}(Unit Erasure Models)
\label{ex:UnitErasure}
Let $\C$ be a bicartesian closed category. Then the functor $\C \ra
\terminal$ is a $\lambda\forall$-fibration. The unique object of
$\terminal$ is a trivial Abelian group object. By taking $\C$ to be
the category of complete partial orders and continuous functions, and
by choosing the flat pointed cpo $\bbQ_{\bot}$ to interpret $\qnt$ we
obtain a model corresponding to Kennedy's unit-erasure model. This
model supports a plethora of primitive operations, including all the
standard arithmetical ones. However, the model also contains many
functions which are not dimensionally invariant --- Kennedy uses
relational parametricity~\cite{reynolds1983types} to remove these
unwanted elements; we will come back to his relational model in
Section~\ref{sec:param}.
\end{example}


\begin{example}(Syntax of Dimension Types)
We can construct a $\lambda D$-model $\ClLD$ from the syntax. The base category $\B$ is the Lawvere theory of Abelian groups $\LAb$, and the fibre $\ClLD_n$ consists of types and (equivalence classes of) terms definable with $n$ dimension variables, where terms are considered modulo conversion. This is a term model and hence provides a completeness result for $\lambda D$-models.
\end{example}


\begin{example}($\Fam\Set$ Model of Dimension Types)
\label{example:famset}
Let $\Fam\Set$ be the category whose objects are pairs
$(I,\{X_i\}_{i\in I})$ of a set $I$ and an $I$-indexed family of sets
$\{X_i\}_{i\in I}$. A morphism $(I,\{X_i\}_{i\in I}) \to
(J,\{Y_j\}_{j\in J})$ is a pair $(f,\{\phi_i\}_{i\in I})$ where $f$ is
a function $f:I\to J$ and $\phi_i$ is a function $\phi_i:X_i\to
Y_{f(i)}$ for all $i\in I$. It is well known that the forgetful
functor $(I,\{X_i\}_{i\in I}) \mapsto I : \Fam\Set\to \Set$, taking a
family to its index set, is a $\lambda\forall$-fibration (see e.g.\
Jacobs~\cite[Lemma 1.9.5]{jacobs1999categorical}).  For any given set
$B$ of fundamental dimensions (e.g.\ $\lengthDim$, $\timeDim$, $\massDim$ etc.), let $G$ be
the free Abelian group on $B$. Suppose that we also have a set $Q_d$
of quantities for each dimension $d \in G$ (for instance, we can
choose $Q_d = \bbQ \times \{ \overline{d} \}$ where $\overline{d}$ is
a unit of measure for the dimension $d$, e.g.\
$\overline{\lengthDim} = \mathrm m$, $\overline{\timeDim} = \mathrm s$,
$\overline{d\cdot d'} = \overline{d}\cdot\overline{d'}$ etc). We then
have a $\lambda D$-model with $\qnt$ interpreted as $(G,\{Q_d\}_{d\in
  G})$.

In this model, a dimension expression $\Dj{\Dvar_1,\dots, \Dvar_n}
\Dim$ is interpreted as a function $G^n\to G$ using the free Abelian
group structure on $G$: for each valuation of the dimension variables
as physical dimensions, we have an interpretation of the expression as
a physical dimension. A type with a free dimension variable $\Tj \Dvar
T$ is interpreted as a family of sets, indexed by the dimensions in
$G$. Similarly a term with a free dimension variable is interpreted as
a family of functions, one for each dimension in $G$. This model does
support many primitive operations, but it does not support dimension
invariant polymorphism. For instance, the model supports adding a term
$\mathsf{eq}:\forall \Dvar_1.\forall \Dvar_2.\bool$ which tests
whether two dimensions are the same, which is clearly not invariant
under change of representation.

Related examples include the relations fibration $\Rel \to \Set$ and
the subobject fibration $\Sub{\Set} \to \Set$. This example can also
be generalised to the fibration $\Fam{\C}\to\Set$, which is a
$\lambda\forall$-fibration if $\C$ is bicartesian closed.

\end{example}


We next look at a class of $\lambda D$ models that all share one thing in common --- the fibres in the $\lambda \forall$ fibration are functors (in a particularly nice way). Interestingly, we can prove a general theorem (Theorem~\ref{thm:BC}) that gives these $\lambda D$ models as simple examples (Example~\ref{ex:MonAct} and Example~\ref{ex:presheaves}).

We first introduce some notation.
Let $\mathcal S$ be a category (e.g.~$\mathcal S=\Set$), and
consider the category $\Cat\sslice\mathcal S$:
The objects are pairs $(C,P:C\to \mathcal S)$ of a small category $C$ and a functor $P$.
The morphisms are also pairs $(F,\phi):(C,P)\to (D,Q)$,
of a functor $F:C\to D$ and a natural transformation
$P\to QF$.
The obvious projection functor $\Cat\sslice \mathcal S\to \Cat$,
with $(C,P)\mapsto C$,
is a fibration. The fibre over a small category $C$ is the category
of functors $[C\to\Set]$, and reindexing is by precomposition of functors.

\paragraph*{A Source of Fibrations with Simple Products}
\begin{theorem}
\label{thm:BC}
If $\mathcal S$ has all small limits
then the fibration $\Cat\sslice \mathcal S\to \Cat$ has simple products.
\end{theorem}
\begin{proof}[Proof notes]
This result appears to be fairly well known in the folklore,
(e.g.~\cite[end of \S 3]{lawvere-adjointness}, \cite{mellies-zeilberger})
but since it is important in what follows we sketch a proof.

For any functor $F:C\to D$,
the reindexing functor $F^*:\mathcal S^D\to \mathcal S^C$
has a right adjoint
$F_*:\mathcal S^C\to\mathcal S^D$, which is called `right Kan extension along $F$',
which always exists when $\mathcal S$ has limits.

For simple products, we are only interested in a right adjoint to weakening,
i.e. in the functor $\forall_C:\mathcal S^{C\times D}\to \mathcal S^C$
which is right Kan extension along
along the projection functor $\pi_C:C\times D\to C$.
Expanding the definitions, we see that $\forall_C(P):C\to\mathcal S$ is
a pointwise limit:
\begin{equation}
(\forall_C\,P)(c)=\lim_{d\in D}P(c,d)\text.
\label{eqn:forall-limit}
\end{equation}
The Beck-Chevalley condition requires that the canonical map
$
F^*\forall_{C'}
\to
\forall_{C}(F\times \id_D)^*
$
is a natural isomorphism for all functors $F:C\to C'$.
Indeed, for any
$P:C\times D\to \mathcal S$, $c\in \mathcal C$:
\begin{multline*}
(F^*(\forall_{C'}\,P))(c)
=
(\forall_C\,P)(F(c))
\cong
\lim_{d\in D}(F(c),d)
=
\lim_{d\in D}(((F\times \id_D)^*(P))(c,d))
\\\cong
(\forall_{C}((F\times \id_D)^*(P))(c)
\text.
\end{multline*}
\end{proof}

\paragraph*{Change of Base}
\label{sec:change-of-base}
In general, a useful way of building fibrations is by changing the base.
If $p:\E\ra \B$ be a fibration,
and $F:\A\ra \B$ is a functor,
then the pullback of $p$ along $F$,
denoted
$F^\ast p : F^\ast \E \rightarrow \A$,
is again a fibration.
\begin{theorem}
\label{thm:change-of-base}
Let $p:\E\ra \B$ be a fibration,
and let $F:\A\ra \B$ be a functor.
\begin{enumerate}
\item If $p$ has simple products and $F$ preserves products,
then $F^\ast p : F^\ast \E \rightarrow \A$ has simple products.
\item If $p$ is bicartesian closed then $F^\ast p: F^\ast \E \rightarrow \A$
is bicartesian closed.
\item If $G$ is an Abelian group object in $\A$ and
  $(p,F(G),Q)$ is a $\lambda D$-model
  then $(F^\ast p,G,(G,Q)$ is also a $\lambda D$-model.
\end{enumerate}
\end{theorem}
\begin{proof}
For item~(1): for any $A \in \A$ reindexing along a projection $\pi_{_{A}}: A \times A' \rightarrow A$ in $\A$ is by construction reindexing along $F(\pi_{_{A}})$ which (as $F$ preserves finite products) is the same as reindexing along a projection $\pi_{_{FA}} : FA \times FA' \rightarrow FA$, which has a right adjoint and satisfies the Beck-Chevalley condition, since $p$ has simple products.

For item~(2): $F^*p$ is a bicartesian closed fibration since each fibre $\fibre{(F^*\E)}{A}$ is by construction of the form $\fibreE{FA}$ and hence bicartesian closed, and reindexing by $f$ in $\A$ is by construction defined to be reindexing by $Ff$ in $\B$ which preserves the structure.

Item~(3) is an immediate corollary.
\end{proof}


Here is a simple illustration of the change of base result:
The unit-erasure fibration $(\C\to 1)$ arises from pulling back the families fibration
$\Fam\C\to\Set$ along the unique product-preserving functor $1\to\Set$.


\begin{example}(A Model Built from Group Actions)
\label{ex:MonAct}
Let $G$ be a group. Recall that a $G$-set consists of a set $A$ together with a function, called a group action, $\cdot_A:G\times A\to A$ such that $e\cdot_{A} a=a$ and $(gh)\cdot_A a=g\cdot_A (h\cdot_A a)$. The category $\GrpSet$ has as objects pairs $(G,A)$ where $G$ is a group and $A$ is a $G$-set. A morphism $(G,A) \rightarrow (H,B)$ in $\GrpSet$ is given by a group homomorphism $\phi:G\rightarrow H$ and a function $f:A \rightarrow B$ such that for any $g\in G$ and $a\in A$ we have $f (g\cdot_A a) = (\phi g)\cdot_B (f a)$.
%From now on we will omit subscripts of group actions where they can be inferred from context.

Let $\Grp$ be the category of groups and homomorphisms, then we call the forgetful functor $\GrpSet\to \Grp$ the $\GrpSet$ fibration.

\begin{proposition}
Let $p:\GrpSet \rightarrow \Grp$ be the $\GrpSet$ fibration, let $G$ be an Abelian group, and $Q$ a $G$-set. Then $(p,G,Q)$ is a $\lambda D$-model.
\end{proposition}

\begin{proof}
For any group $G$ the fibre above $G$ is the category of $G$-sets for fixed $G$.
This is isomorphic to the
the functor category ${\Set}^G$,
where we consider the group $G$ as a category with one object $*$
and $G$ as the set of morphisms.
Indeed, there is a product-preserving, full and faithful functor $\Grp\to\Cat$,
taking a group to the corresponding one-object category.
The fibration $\GrpSet\to\Grp$ is thus the pullback of
the fibration $\CatSet\to\Cat$ along this embedding $\Grp \to \Cat$.
Thus, by Theorem~\ref{thm:BC} and Theorem~\ref{thm:change-of-base},
$\GrpSet\to\Grp$ has simple products.

Each fibre is bicartesian closed.
The products, coproducts and function spaces are inherited from $\Set$.
For the function space, let $A$ and $B$ be $G$-sets;
then the set of functions $(A\to B)$ is also a $G$-set,
with $(g\cdot_{(A\to B)}f)(x)\coloneqq g\cdot_B(g^{-1}\cdot_A x)$.
It follows that reindexing preserves the bicartesian closed structure.

A group can be given the structure of an Abelian group object in $\Grp$ if it is an Abelian group. (Indeed, an Abelian group object in $\Grp$ is the same thing as an Abelian group). Hence $(p,G,Q)$ is a $\lambda D$-model.
\end{proof}

In the $\GrpSet$ fibration, we model a type with a free dimension variable, $\Tj\Dvar T$, as a $G$-set, and a term with a free dimension variable as a function that is invariant under $G$. We explore this model in more detail in Section~\ref{sec:MonSet}.
\end{example}

More generally, instead of having sets and group actions, we also have $\lambda D$-models built from actions of groupoids.

\begin{example}(A Model Built from Groupoid-actions)
\label{ex:presheaves}
Recall that a \emph{groupoid} is a small category $C$ where every morphism is an isomorphism, and that a functor $C\to\Set$ is called a groupoid action (or presheaf). The category $\GpdSet$ has as objects pairs $(\A,\phi)$ where $\A$ is a groupoid and $\phi:\A\to\Set$ is a functor. A morphism $(\A,\phi) \rightarrow (\B,\psi)$ in $\GpdSet$ is given by a functor $F:\A\rightarrow \B$ and a natural transformation $\eta:\phi \Rightarrow \psi \cdot F$ between functors $\A \to\Set$.

Let $\Gpd$ be the category of groupoids and functors. Then the forgetful functor $\GpdSet\to \Gpd$, which we call the $\GpdSet$ fibration, is a $\lambda \forall$-model.
The proof of this is very similar to Example~\ref{ex:MonAct}.

The $\GpdSet$ fibration is related to the other fibrations by change of base.
\begin{itemize}
\item
The families fibration $\Fam\Set\to\Set$ arises from pulling back the groupoid-action fibration
$\GpdSet\to\Gpd$ along the discrete-groupoid-functor $\Set\to\Gpd$.
\item The group-action fibration $\GrpSet\to\Grp$ arises from pulling
  back the groupoid-action fibration $\GpdSet\to\Gpd$ along the functor
  $\Grp\to\Gpd$ that regards each group as a groupoid with one
  object.
\end{itemize}

We now discuss $\lambda D$-models in $\GpdSet$.

Let $f:G\to H$ be a homomorphism of Abelian groups. This induces a groupoid
whose objects are elements of $H$ and where the hom-sets are $\mathrm{mor}(h,h')=\{g\in G~|~f(g)\cdot_Hh=h'\}$. The group operation in $G$ provides
composition of morphisms.
This groupoid can be given the structure of an Abelian group object in
$\Gpd$, and, moreover, every Abelian group in $\Gpd$ arises in this way~\cite{brown-spencer}.

We have already seen that the $\GpdSet$ fibration subsumes the families and
group-actions fibrations. It also subsumes them as $\lambda D$-models.
To recover group-actions (Ex.~\ref{ex:MonAct}),
let $G$ be an Abelian group of scale factors.
The Abelian group object induced by
the unique homomorphism $G\to 1$ is a one-object groupoid,
and hence we build
the $\lambda D$-models of group actions.
To recover the familes example (Ex.~\ref{example:famset}),
fix a set of dimension constants and let $H$ be the
free Abelian group on that set. The unique homomorphism $1\to H$ induces the
discrete groupoid whose objects are $H$, and hence we build the $\lambda D$-%
models of families of sets.
We can also combine the two approaches
by considering the zero homomorphism $G\to H$.
\end{example}

% \begin{example}\label{ex:fibrations}
% A third example in this class is given by the codomain fibration $cod: \Fib \rightarrow \Cat$, which has for any category $\A$, the fibre $\fibre{\Fib}{\A}$ consists of fibrations $\E \rightarrow \A$ for any category $\E$. To use Theorem~\ref{thm:BC} we must show that the fibres can be equivalently seen as functor categories $\mathcal{D}^{\SqFun[\A]}$ for a fixed category $\D$. By the Grothendiek construction we have that fibrations $\E \rightarrow \A$ are equivalent to functors $\A^{op} \rightarrow \Cat$. So by taking $\D$ to be $\Cat$ and the functor $\SqFun[\blank]:\Cat \rightarrow \Cat$ to be defined by $\SqFun[\A] = \A^{op}$ on objects, and $\SqFun[F] = F^{op} = F$ on morphisms, we are in the situation covered by Theorem~\ref{thm:BC}. Hence, by using similar arguments to Example~\ref{ex:MonAct} and Example~\ref{ex:presheaves}, we can show that $cod:\Fib \rightarrow \Cat$ is a $\lambda \forall$-model.
% \note{Is it interesting, though? Should it be included?}
% \end{example}


%\begin{example}
%\label{ex:uomLaw}
%Let $G$ be an Abelian group and $\phi$ a $G$-set. Define a fibration $\msf{U_G}:\E \ra \LAb$ to have as fibres $\fibreE{n}$ the functor category $Set^{G^n}$. Then the triple $(U_G, \terminal, \phi)$ is a $\lambda D$-model. We again defer the proof to Section~\ref{sec:MonSet}.
%\end{example}

% Note that in the $G$-set examples, there is a canonical choice for the $G$-set used to interpret $\qnt$, namely the $G$-set $G \times G \ra G$ (or $\terminal \times \terminal \rightarrow \terminal$ above) given by multiplication in the group. When we consider $G$-sets as functors $G \ra \Set$, this $G$-set is simply the hom-functor which, via Yoneda, makes it a particularly interesting $G$-set to choose. We will expand on this in section~\ref{sec:MonSet}.
% \note{expand on this comment or remove.}




% The key lemma is.....
% \begin{lemma}
% Suppose that $X,Y$ are categories and the projection map is denoted $\pi : X\times Y \rightarrow Y$. If $\phi : X \times Y \rightarrow Set$ then
%  \[
%   (Lan_\pi \phi) x = \int^{y\in Y} \phi (x, y)
%  \]
% \end{lemma}
% \begin{proof}
% \begin{align*}
%  (Lan_\pi \phi )x_0 &\cong  \int^{(x,y)\in X \times Y} X(\pi (x,y),x_0) \times \phi (x,y)\\
% 		    &\cong \int^{y\in Y} \int^{x \in X} X(x,x_0) \times \phi (x,y) \text{ Fubini}\\
% 		    &\cong \int^{y\in Y} \phi (x_0,y) \text{ by Yoneda for Ends}
% \end{align*}
% \end{proof}
%
% \begin{lemma}(Yoneda for Ends)
%  \[
% \int^{x \in X} X(x,x_0) \times \phi (x,y) \cong \phi(x_0,y)
% \]
% \end{lemma}
% \begin{proof}
% \begin{align*}
%  Set(\int^{x \in X} X(x,x_0) \times \phi (x,y), S) & \cong \int_{x\in X} Set(X(x,x_0) \times \phi(x,y),S) \text{ by preservation of limits}\\
%  & \cong \int_{x\in X} Set(X(x,x_0), Set(\phi (x,y), S))\\
%  & \cong Nat(X(\blank ,x_0), Set(\phi (\blank ,y) ,KS)) \text{ by Yoneda}\\
%  & \cong Set(\phi(x_0,y), S)
% \end{align*}
% Which implies the result.
% \end{proof}


%\noindent The examples that are related by change-of-base are:
%\begin{itemize}
%\end{itemize}


% Choosing an internal Abelian group in a category $\C$ is equivalent to giving a finite product-preserving functor from the
% Lawvere theory of Abelian groups $\LAb$ into the category $\C$.
% Thus, as another instance of the change-of-base theorem:
%
% %Given that we have essentially the
% % Kleisli category of $\msf{Ab}$ and its Eilenberg-Moore category
% % arising as bases of UoM-fibrations, it is tempting to wonder if any
% % resolution of the monad $T_{\msf{Ab}}$ can act as the base of a
% % UoM-fibration? However we - for once - resist temptation.
%
% \begin{corollary}
%   Any $\lambda D$-model
%   can be converted into a model with base category $\LAb$, by pulling back.
% \end{corollary}
% \note{I've kept this corollary, but I am not sure why it is interesting.}
%Hidden proof since it is trivial.
%\begin{proof}
%Let $(p:\E \ra \B, G,Q)$ be a UoM-fibration. The fact that $G$ is an internal Abelian group object %means that there is a finite
%roduct-preserving functor $F:\LAb \ra \B$ mapping $\terminal$ to $G$. By change of base, the functor $%F^*p : F^\ast \E \ra \LAb$ is a fibration. As above, $\terminal$ is exponentiable and is by constructi%on an internal Abelian group object. Finally, since $F$ maps $\terminal$ to $G$, the object
%$(\terminal,Q)$ lives above $\terminal$ in $F^\ast p$ and hence $F^\ast p$ has the structure of a UoM-fibration with base $\LAb$.
%\end{proof}

%This is similar to Bob's Theorem - in the statement below I've changed Bob's pullback
%condition to finite product preservation. Finite product preservation
%implies the pullback condition while the pullback condition together
%with preservation of terminal objects implies finite product
%preservation. The examples seem to be finite product-preserving and
%this condition links in nicely with Lawvere theories so I find it
%preferable currently. Bob's theorem (with my modification) is the following
%
%\begin{theorem}[Bob's Theorem]
%\label{thm:bob}
%  Let $q: \D \ra \C$ be a $\lambda_1$ fibration where every object is
%  exponentiable and let $R:L_{\msf{Ab}} \ra \C$ be a finite product
%  preserving functor. Choose an object $X$ of $\C$ and $P$ of $\D$ above $R1 \times X
%  \times X$. Next
%  construct the pullback
%\begin{center}
%\hspace{0.1in}
%\xymatrix{ \E \ar[d]_p  \ar[r]  & & \D\ar[d]^{q}\\
%L_{\msf{Ab}} \times \C \ar[r]_{R \times \Delta} & \C^3 \ar[r]_{\times} & \C }
%\end{center}
%Then $\msf{fst \cdot p} :\E \ra L_{\msf{Ab}}$ together with $R1$ and
%$P$ form a \UoMFibration.
%\end{theorem}





\section{Group Actions and Dimension Types}\label{sec:MonSet}
\note{Add something here about signatures}
\note{Add that this can all be done for models that fit the Theorem~\ref{thm:BC} model}

\noindent In this section we will look in greater detail at the $\lambda D$-model given by the $\GrpSet$ fibration, and so we spell out the reindexing and simple product structure.


Suppose that $\phi: \mathcal{G}\rightarrow \Set$ is an $G$-set, then reindexing along $\pi : G \times H \rightarrow G$ yields the $G \times H$-set given by $\phi \circ \pi$. Or in other words the $G \times H$-set with underlying carrier given by $A$ and the action given by $(m,g) \act{\pi^{\ast}A} x = m \act{A} x$.

Now suppose that $\psi : \mathcal{G} \times \mathcal{H} \rightarrow\Set $ is a $G \times H$-set. According to Theorem~\ref{thm:BC} (equation~\eqref{eqn:forall-limit}),
the underlying set of $ (\forall _\pi \psi)$ is given by $\lim_{y\in \mathcal{H}} \psi (\ast, y)$. If we use the universal property of limits to note that
\[
 \lim_{y\in \mathcal{H}} \psi (\ast, y) \cong \Set \bbracket{1, \lim_{y\in \mathcal{H}} \psi (\ast, y)} \cong Nat \bbracket{K1, \psi (\ast, \blank)}
\]
then we can see that $|\forall_\pi \psi | = \{y \in |\psi| \; \; | \; \; \forall h\in H \; \; \psi(e_A, h) y = y \}$, and the action on morphisms is given by $(\forall_\pi \psi) g \; x = \psi (g, e_H) x$.

Notice that to give the action of $\forall_\pi \psi$ on morphisms we had to make a particular choice of morphism in $H$, namely the identity element $e_H$. However, any element of $H$ would have given the same result since
\[
  \psi (g, h) x = \psi (g, e_H) \cdot \psi (e_G, h) x = \psi (g, e_H) x
\]


A particularly nice property about this model, is that many of the properties of \UoM that Kennedy proves using parametricity, can be shown to hold in the $\GrpSet$-fibration, without having to define a separate relational semantics. The downside of not giving a separate relational semantics is that we do not capture the complete essence of parametricity, just a useful portion of it. We first note three semantic isomorphisms. The first is a substitution lemma, which holds in any model.

\begin{lemma}(Substitution Lemma)
\label{lem:subst}
Suppose that $\Delta, X \vdash T \; \msf{ Type}$ and that $\Dj\Delta \Dim$ denotes a dimension expression, then
\[
 \sem{T[D/X]} \cong (id_{\sem{\Delta}} , \sem{D})^\ast \sem{T}
\]
\end{lemma}
\begin{proof}
 By induction on the structure of $T$.
\end{proof}


Explicitly, Lemma~\ref{lem:subst} says that the semantics of substituting a dimension expression for a dimension variable is given by reindexing along the identity paired with the dimension expression. Since reindexing is given by precomposition we have that
\[
 (id_{\sem{\Delta}} , \sem{D})^\ast \sem{T} \cong \sem{T}(id_{\sem{\Delta}}, \sem{D})
\]
Or in other words, substitution of the $n^{th}$ unit variable is given by precomposition at the $n^{th}$ component.

For the rest of this section, we will use semantic brackets $\sem{\, \underline{\hspace{2mm}}\, }$ to refer only to the $\GrpSet$ interpretation.

\begin{lemma}
\label{lemma:allarrows}
 Suppose that $\Delta, u \vdash S,T \; \msf{ Type}$, then
\[
|\sem{\forall u. S \rightarrow T}| \cong \GSet (\sem{S}(\underbrace{\ast,...,\ast}_{n-\text{times}}, \blank),\sem{T}(\underbrace{\ast,...,\ast}_{n-\text{times}}, \blank))
\]
\end{lemma}
\begin{proof}
By the Kan extension formula and Yoneda.
\end{proof}

This Lemma says that in the $\GrpSet$ model a universally quantified variable over an arrow type, can be considered as a natural transformation between the domain and codomain of the arrow type, with the first $n$ components fixed. Finally, we note a useful corollary of Lemma~\ref{lemma:allarrows}.
\begin{corollary}
\label{cor:subs}
Suppose that $\Delta, X \vdash T \; \msf{ Type}$, then
 \[
  |\sem{\forall X . \qnt(X) \rightarrow T}| \cong |\sem{T[1/X]}|
\]
\end{corollary}
\begin{proof}
 By Lemma~\ref{lemma:allarrows}, Lemma~\ref{lem:subst} and Yoneda.
\end{proof}



We now prove some theorems about the $\GrpSet$ fibration that are akin to the parametricity results of Kennedy's original paper. The proofs here involve applications of Lemma~\ref{lem:subst}, Lemma~\ref{lemma:allarrows} and Corollary~\ref{cor:subs}. Firstly, lets have a look at the interplay between scaling factors and polymorphic functions.

\begin{theorem}(Scaling Factors)
\label{thm:ScalFact}
Let $t$ be a closed term of type $\forall X. \qnt(X) \rightarrow \qnt(X^n)$, where $n\in \mathbb{N}$. Then for all $g \in G$ and $x \in |\sem{\qnt(X)}|$,
\[
\sem{t} (g \cdot x) = g^n \cdot \sem{t} x
\]
\end{theorem}
\begin{proof}
We know from Lemma~\ref{lemma:allarrows} that
\[
\sem{t} \in \GSet(|\sem{\qnt(X)}|,|\sem{\qnt(X^n)}|)
\]
% This means that $\sem{t}$ must satisfy the diagram
% \begin{diagram}
% |\qnt(u)|       &\rTo^{\sem{t}}        &|\qnt(u^n)| \\
% \dTo<{\qnt(u)g}     &                      &\dTo>{\qnt(u^n)m } \\
% |\qnt(u)|        &\rTo_{\sem{t}}        &|\qnt(u^n)| \\
% \end{diagram}
Or in other words $\sem{t} (g \cdot x) = g^n \cdot \sem{t} x$, for all $x \in |\sem{\qnt(X)}|$ as required.
\end{proof}
This theorem tells us that polymorphic functions are \emph{invariant under scaling}. Intuitively we see that scaling factors must be changed in an appropriately polymorphic way.

If we apply Corollary~\ref{cor:subs} to the type $\forall X. \qnt(X)\rightarrow \qnt(X^n)$ we see that
\[
 |\sem{\forall X. \qnt(X)\rightarrow \qnt(X^n)}| \cong |\sem{\qnt(1^n)}| \cong |\sem{\qnt(1)}| \cong G
\]
This implies that terms of type $\forall X. \qnt(X)\rightarrow \qnt(X^n)$ vary only by scalar multiplication.


Next, we wish to prove that the type  $\forall X . \qnt(X^2) \rightarrow \qnt(X)$ is uninhabited.

\begin{theorem}
\label{thm:UninhabType}
 There is no term of type $\forall X . \qnt(X^2) \rightarrow \qnt(X)$., i.e., we cannot write a polymorphic square root function.
\end{theorem}
\begin{proof}
To see this we show that there exists a model where the existence of such a term is impossible, for simplicity we assume that the typing context is empty. Consider the $\lambda D$-model $(p:\GrpSet \rightarrow \Grp, \mathbb{Z}_2, \mathbb{Z}_2)$  consisting of the $\GrpSet$ fibration, and the Abelian group $\mathbb{Z}_2 = \{ -1, 1\}$ as the exponentiable Abelian group object and the $Q$ object. Lemma~\ref{lemma:allarrows} says that the interpretation of the type $\forall X . \qnt(X^2) \rightarrow \qnt(X)$ is given by
\begin{align*}
  |\sem{\forall X . \qnt(X^2) \rightarrow \qnt(X)}|  &  \cong \GroupSet{\mathbb{Z}_2} (|\sem{\qnt (X^2)}|, |\sem{\qnt(X)}|)
\end{align*}
Or in other words, any element $f$ of $ |\sem{\forall X . \qnt(X^2) \rightarrow \qnt(X)}|$, satisfies for all $g, x \in \bbZ_2$
\[
f (g^2 \cdot x) = g \cdot (fx) \; \; \; \; (\ast)
\]
If $f$ exists, then either $f(-1) = -1$ or $f(-1) = 1$, but both lead to contradictions. Suppose that $f(-1) = -1$, then by $(\ast)$
\[
 f((-1)^2 \cdot -1) = (-1) \cdot f(-1)
\]
But the left-hand side is equal to $-1$ and the right-hand side is equal to $1$. A similar argument shows that $f(-1)=1$ is also not possible, and hence there exists no non-trivial $f$, as required.
\end{proof}

Next, we can prove a theorem that relates a dimensionally invariant function to a dimensionless one. This is a simplified version of the \emph{Buckingham Pi Theorem} of dimensional analysis, which first appeared here \cite{buckingham1914physically} but a more modern introduction can be found here \cite{sonin2001physical}.

\begin{theorem}
 \label{lem:AppSubs}
Consider the $\GrpSet$ semantics for $\lambda D$-model. Then
\[
 |\sem{\forall X . \qnt(X) \rightarrow \qnt(X) \rightarrow \qnt(1)}| \cong |\sem{\qnt(1)\rightarrow \qnt(1)}|
\]
\end{theorem}
\begin{proof}
We obtain this result as a direct consequence of Corollary~\ref{cor:subs}.
\end{proof}

Finally, we prove a result about the higher order type $\forall X_1. \forall X_2. (\qnt(X_1) \rightarrow \qnt(X_2)) \rightarrow \qnt(X_1 \cdot X_2)$.

\begin{theorem}\label{thm:UninhabInt}
There exists no non-trivial term of type $\forall X_1. \forall X_2. (\qnt(X_1) \rightarrow \qnt(X_2)) \rightarrow \qnt(X_1 \cdot X_2)$.
\end{theorem}
\begin{proof}
Set $G$ to be $\bbZ_2$ and similarly $Q$ to be $\bbZ_2$ as well. Then interpreting this type we have that
 \[
\sem{\forall X_1. \forall X_2. (\qnt(X_1) \rightarrow \qnt(X_2)) \rightarrow \qnt(X_1 \cdot X_2)} = \{ t \in (Q \rightarrow Q) \rightarrow Q \; | \; \forall g_1,  g_2 \in G \; \; (g_1 g_2) \cdot t = t \}
 \]
Hence for any $t \in  \sem{\forall X_1. \forall X_2. (\qnt(X_1) \rightarrow \qnt(X_2)) \rightarrow \qnt(X_1 \cdot X_2)}$ and $f \in Q \rightarrow Q$, we have that
\[
(g_1 g_2) \cdot tf = tf
\]
which is not possible. Since for any value of $tf$, let $g_1 = 1$ and $g_2 = -1$, then you reach a contradiction.
\end{proof}


% \note{****Add to other document that it is easy to generate ``free theorems'' in MonSet fibration and there are many examples especially when playing around with \qnt(X) and arrows}










\section{Relational Models}
\label{sec:param}
In Section~\ref{sec:MonSet} we pointed out that many of the results that we proved in the $\GrpSet$ $\lambda D$-model are results that  Kennedy proves in \cite{Kennedy:1997:RPU:263699.263761} using parametricity. It's curious how the parametricity-style proofs in the $\GrpSet$ $\lambda D$-model are simple and slick and do not require a separate relational semantics. One can't help but wonder, is the $\GrpSet$ $\lambda D$-model really as good as having full-blown parametricity at your finger tips?

To answer this question we look at a general method of attaching a (fibrational) logic to a $\lambda D$-model to give a notion of a \emph{relational} $\lambda D$-model . This allows us to reconstruct Kennedy's relational parametricity in our setting (Example~\ref{ex:relKen}), as well as talking about a relational version of the $\GrpSet$ $\lambda D$-model (Example~\ref{ex:relMon}). These two examples are intrinsically linked, which we make precise in Theorem~\ref{thm:closed-prog-equiv}.

To begin this section, we must first recall a theorem about the composition of fibred structure.

\vspace{4mm}
\noindent
\begin{minipage}[l]{0.75\linewidth}
\begin{theorem}
\label{thm:CompOfProd}
Suppose that $p:\A \rightarrow \B$ and $q:\B \rightarrow \mathcal{\C}$ are fibrations and that $u:\A \rightarrow \mathcal{\C}$ denotes the composite $q \circ p$ (and hence is also a fibration). Suppose further that $q$ has simple products, and for any projection map $\pi : X \times Y \rightarrow X$ in $\mathcal{\C}$ we denote the Cartesian morphism in $\B$ above it by $\pi^\S_B : \pi^\ast B \rightarrow B$. Then $u$ has simple products that are preserved by $p$ if and only if for any projection map $\pi : X \times Y \rightarrow Y$ in $\mathcal{C}$, the functor $(\pi^\S_B)^\ast : \A_B \rightarrow \fibre{\A}{\pi^\ast B}$ has right adjoints for all $B \in \B_{\C}$, satisfying the Beck-Chevalley condition.
\end{theorem}
\end{minipage}
\begin{minipage}{0.25\textwidth}
\vspace{-12mm}
\begin{diagram}
\A     &\rTo^{p}        &\B \\
       &\rdTo_{u}      &\dTo>{q}\\
       &                &\mathcal{\C}\\
\end{diagram}
 \end{minipage}
\begin{proof}
This theorem is proven by using the factorisation and lifting properties of the 2-category $Fib$ as outlined in \cite{hermida1999some}. Though the proof is not too difficult it does require 2-categorical technology, which we do not introduce here. Hence, we leave the proof as an exercise for the 2-category-savvy reader.
\end{proof}

\noindent
\begin{minipage}[l]{0.75\linewidth}
Now to use this theorem. Given a $\lambda D$-model and a logic, there is a natural way to glue them together to provide a relational semantics.

Let $(G,\, Q_0,\, u:\A \rightarrow \mathcal{L})$ be a $\lambda D$-model, $F:\A \rightarrow \B$ a product preserving functor and $p: \E \rightarrow \B$ a Cartesian closed fibration with products. Consider the pullback of $p$ along $F$, and let $Q_R$ denote an object in the fibre $\fibreE{F(Q_0)}$.
 \end{minipage}
\begin{minipage}{0.25\textwidth}
\vspace{-0mm}
\begin{diagram}
  F^\ast (\E)	\SEpbk	&\rTo		&\E \\
  \dTo<{F^\ast p}	&		&\dTo>{p}\\
  \A			&\rTo_{F}	&\B\\
 \end{diagram}
 \end{minipage}


\begin{theorem}\label{thm:rel-model-generator}
$(G,(Q_0,Q_R), F^\ast p \circ u :F^{\ast} \E \rightarrow \mathcal{L})$ as constructed above, is a $\lambda D$-model.
\end{theorem}

\begin{proof}
Clearly $G$ is an Abelian group object in $\mathcal{L}$, and $(Q_0, Q_R)$ is in the fibre $\fibre{F^\ast \E}{G}$. To check that $F^\ast p \circ u$ is a Cartesian closed fibration is a simple exercise that we leave to the reader. And to see that $F^\ast p \circ u$ has simple products we first note that since $p$ has all products, so does $F^\ast p$. Hence, $F^\ast p \circ u$ has simple products by Theorem~\ref{thm:CompOfProd}.
\end{proof}

\begin{example}\label{ex:relKen}
  This example uses Theorem \ref{thm:rel-model-generator} to generate
  Kennedy's original relationally parametric model of dimension types
  \cite{Kennedy:1997:RPU:263699.263761} from essentially the dimension
  erasure model back in Example \ref{ex:UnitErasure}. Let $G$ be an
  Abelian group. Then using the notation from above, let $\mathcal{L}$
  be the Lawvere theory of Abelian groups $\LAb$, $\A$ be the category
  $\LAb \times \Set$, $u: \LAb \times \Set \rightarrow \LAb$ be the
  fibration given by the first projection, and
  $p:\Sub{\Set}\rightarrow \Set$ be the subset fibration. Define
  $F:\LAb \times \Set \rightarrow \Set$ to be the product preserving
  functor defined on objects $(n,X) \in \LAb \times \Set$ by $F(n, X)
  = G^{n} \times X \times X$, and on morphisms $(f, g) : (n,X)
  \rightarrow (m, Y)$ by $F(f,g) = (G^f, g, g)$. Finally we let $Q_0 =
  G$, and $Q_R = \{(g,g_1,g_2) \mathrel| gg_1 = g_2 \} \subseteq G
  \times G \times G$.

  In this model, each type $\Delta \vdash T$ is interpreted as a
  triple $(|\Delta|, \semo{T}, \semr{T}) \in \LAb \times \Sub{\Set}$,
  where $\semr{T} \subseteq G^n \times \semo{T} \times
  \semo{T}$. Spelling this out explicitly, we have the following
  interpretations, which are equivalent to Kennedy's original
  relationally parametric model for dimension types:
  \begin{displaymath}
    \begin{array}{l@{\hspace{0.3em}}c@{\hspace{0.3em}}l}
      \semr{\Delta \vdash \qnt(D)} & = & (|\Delta|, G, \{ (g, g_1, g_2) \mathrel| (\sem{D}g)g_1 = g_2 \}) \\
      \semr{\Delta \vdash T \times U} & = &
      (|\Delta|,
      \begin{array}[t]{@{}l}
        \semo{T} \times \semo{U}, \\
        \{ (g, (t_1,u_1), (t_2,u_2)) \mathrel| (g,t_1,t_2) \in \semr{T}, (g,u_1,u_2) \in \semr{U} \}
      \end{array}
      \\
      \semr{\Delta \vdash T \to U} & = &
      (|\Delta|,
      \begin{array}[t]{@{}l}
        \semo{T} \to \semo{U}, \\
        \{ (g, f_1, f_2) \mathrel| \forall t_1, t_2.~(g,t_1,t_2) \in \semr{T} \implies (g,f_1t_1, f_2t_2) \in \semr{U} \}
      \end{array}
      \\
      \semr{\Delta \vdash \forall X.~T} & = & (|\Delta|,
      \begin{array}[t]{@{}l}
        \semo{T},
        \{ (g, t_1, t_2) \mathrel| \forall g' \in G.~((g,g'), t_1, t_2) \in \semr{T} \} )
      \end{array}
    \end{array}
  \end{displaymath}
  Note that, in the interpretation types $\forall X.~T$, the
  ``carrier'' (i.e., the second component) is exactly the carrier of
  the interpretation of $T$.
\end{example}


\begin{example}\label{ex:relMon}
  We can also apply Theorem \ref{thm:rel-model-generator} to obtain a
  natural relational model for the $\GrpSet$ $\lambda D$-model. As
  before, let $G$ be an Abelian group and $\mathcal{L}$ be the Lawvere
  theory of Abelian groups $\LAb$. Let $\A$ be the category
  $\bigcup_{n\in \LAb} (\GroupSet{G^{n}})$, $u: \A \rightarrow \LAb$
  be the fibration with each fibre above $n$ given by the category
  $\GroupSet{G^{n}}$ and $p:\Sub{\Set} \rightarrow \Set$ be the subset
  fibration. Define $F:\bigcup_{n\in \LAb} (\GroupSet{G^{n}})
  \rightarrow \Set$ to be the product preserving functor defined on
  objects $(X,\phi) \in \GroupSet{G^{n}}$ by $F(X, \phi) = G^{n}
  \times X \times X$ and on morphisms $(f, \alpha) : (X,\phi)
  \rightarrow (Y, \psi)$, where $(X,\phi) \in \GroupSet{G^{n}}$,
  $(Y,\psi) \in \GroupSet{G^{m}}$, $f: X \rightarrow Y$ and $\alpha :
  G^n \rightarrow G^m$ such that $\psi (\alpha, f) = f \phi$, by
  $F(f,\alpha) = (\alpha, f, f)$. Finally, we let $Q_0 = (G,\phi)$,
  where $\phi$ denotes group multiplication, and $Q_R = \{(g,g_1,g_2)
  \mathrel| gg_1 = g_2 \} \subseteq G \times G \times G$.

  Then each type $\Delta \vdash T$ is again interpreted as a triple
  $(|\Delta|, \semo{T}, \semr{T}) \in \LAb \times \Sub{\Set}$, with
  $\semr{T} \subseteq G^n \times \semo{T} \times \semo{T}$. The only
  difference between the interpretation of types in this example and
  Example~\ref{ex:relKen} is the second component of the
  interpretation of dimension quantification:
  \begin{displaymath}
    \begin{array}{l@{\hspace{0.3em}}c@{\hspace{0.3em}}l}
      \semr{\Delta \vdash \forall X.~T} & = & (|\Delta|,
      \begin{array}[t]{@{}l}
        \{ t \in |\semr{T}| \mathrel| \forall g \in G.~((e_{G^{|\Delta|}},g), t, t) \in \semr{T} \},\\
        \{ (g, t_1, t_2) \mathrel| \forall g' \in G.~((g,g'), t_1, t_2) \in \semr{T} \} \}
      \end{array}
    \end{array}
  \end{displaymath}
  This interpretation, in contrast to the interpretation in Example
  \ref{ex:relKen}, has ``cut-down'' the carrier of the interpretation
  of $\forall$-types to only include the ``parametric'' elements. As a
  consequence, this interpretation satisfies an analogue of the
  \emph{Identity Extension} lemma from relationally parametric models
  of System F \cite{reynolds1983types}. For all type interpretations
  $(|\Delta|, \semo{T}, \semr{T})$, we have:
  \begin{displaymath}
    \forall x_1, x_2 \in \semo{T}.~(e,x_1,x_2) \in \semr{T} \Leftrightarrow x_1 = x_2
  \end{displaymath}
  Compare this to the identity extension property for System F models,
  which states that if we instantiate the relational interpretation of
  a type with the equality relation for all of its free variables,
  then the resulting relation is the equality relation. In the current
  setting, equality relations for the free variables are replaced by
  the unit element of the groups $G^{|\Delta|}$. Indeed, this model is
  equivalent to the restriction to one-dimensional scalings of the
  reflexive graph model for System F$\omega$ with geometric symmetries
  presented by Atkey \cite{atkey14conservation}.
\end{example}

We end this discussion of relational models by showing the
relationships between the models in Examples \ref{ex:relKen} and
\ref{ex:relMon} and the $\GrpSet$ model we considered in detail in
Section \ref{sec:MonSet}. By construction, the carriers of the
interpretations of each type in the model in Example \ref{ex:relMon}
and the $\GrpSet$ model are identical. Moreover, the relational
interpretation in Example \ref{ex:relMon} and the group action in the
$\GrpSet$ model are related in the obvious way:
\begin{theorem}\label{thm:grp-rel-related}
  For all types $\Delta \vdash T~\msf{Type}$, if the interpretation of
  $T$ in the model of Example \ref{ex:relMon} is $(|\Delta|, A, P
  \subseteq G^{|\Delta|} \times A \times A)$ and the $\GrpSet$ model
  interpretation is $(G^n, A, \psi)$, then $(g, a_1, a_2) \in P
  \Leftrightarrow \psi(g,a_1) = a_2$.
\end{theorem}
\begin{proof}
  By induction on the derivation of $\Delta \vdash T~\msf{Type}$.
\end{proof}

Using Theorem \ref{thm:grp-rel-related}, we can see that we could have
used the relationally parametric model to derive the results in
Section \ref{sec:MonSet}. There is literally no difference between the
two models, for the purposes of interpreting the types of our
calculus.

It remains to discuss the relationship between Kennedy's original
relational model (Example \ref{ex:relKen}), and the relational model
in Example \ref{ex:relMon}, that satisfies identity extension. As
noted above, the difference between these interpretations lies in the
different interpretations of the $\forall$-type: Kennedy's model does
not restrict the carrier of the interpretation to just the
``parametric'' elements, \emph{i.e.}, the elements that preserve all
relations. Therefore, the interpretations of types that contain nested
$\forall$s are not directly comparable. However, we can construct a
logical relation between the two interpretations to show the following
property:

\begin{theorem}\label{thm:closed-prog-equiv}
  For any closed term $\vdash t : \msf{bool}$, the interpretation of
  $t$ in Kennedy's model is equal to the interpretation of $t$ in the
  model of Example \ref{ex:relMon}.
\end{theorem}

By the compositionality of the semantics, this means that if we can
show that two open terms $s$ and $t$ are equal in the model of Example
\ref{ex:relMon} (and equivalently, the $\GrpSet$ model), then they
will be contextually equivalent for Kennedy's model.

\note{This also holds between the unit-erasure semantics and the one
  of Example \ref{ex:relMon}. Not sure how to fit this in?}

\section{Conclusions and Future Work} \label{sec:con}
What's so special about G-Sets?

\paragraph*{Acknowledegements.}
We are grateful to participants in a discussion on the categories mailing list
about the $\GrpSet$ fibration.


% \section{Parametricity for Units of Measure}
% Having studied the syntax of Units of Measure categorically, we now
% wish to do the same for parametricity for Units of Measure. But before
% we do that, we ought to spell out what we mean by a parametric
% model. In recent unpublished work on parametricity for \SystemF, we
% took a relationally parametric model of System $F$ to consists of
% \begin{itemize}
% \item For every type $T$ in $n$-free type variables, a functor
%   $\sem{T}_0:|\sets|^n \ra \sets$ together with a lifting
%  $\sem{T}_1:|\rel|^n \ra \rel$.
% \item For every term $\Delta, \Gamma \vdash t : T$, a natural
%   transformation $\sem{t}_0 : \sem{\Gamma}_0 \ra \sem{T}_0$ and
% another natural transformation
% $\sem{t}_1 : \sem{\Gamma}_1 \ra \sem{T}_1$.
% Here $\sem{\Gamma}_0$ and $\sem{\Gamma}_1$ are defined
% using products in $\sets$ and $\rel$ and the interpretations of their
% constituent types.
% \item The identity extension lemma is formulated as $\sem{T}_1
%   \msf{|Eq|}^n = \msf{Eq} \sem{T}_0$. Here $\msf{Eq}:\sets \ra \rel$
%   assigns to each set, the equality relation on it and $\msf{|Eq|}^n$
%   is its obvious extension to $|\sets |^n \ra |\rel |^n$. The
%   fundamental theorem of logical relations states that $\sem{t}_1$ is
%   over $\sem{t}_0 \times \sem{t}_0$, that is $U \sem{t}_1 = (\sem{t}_0
%   \times \sem{t}_0) |U|^n$ where $U:\sets \ra \sets \times \sets$ is
%   the forgetful functor and similarly for $|U|^n$.\text{ by lemma ~\ref{lemma:RanAlongPrj}}
% \end{itemize}
% In pictures, we might visualise the following soothing diagrams
% %\begin{center}
% \[
% \begin{array}{ll}
% \hspace{0.1in}
% \xymatrix{ |\rel |^n \ar[d]_{|U^n|} \ar[r]^{\sem{T}_1} &\msf{\Rel}\ar[d]^{U}\\
%   |\sets |^n \times |\sets|^n \ar[r]_{\sem{T}_0}	&\sets \times \sets}
% %\end{center}\text{ by lemma ~\ref{lemma:RanAlongPrj}}
% \;\;\;\;\;\;\;\;\; &
% \xymatrix{ |\rel |^n \ar[r]^{\sem{T}_1} &\msf{\Rel}\\
%   |\sets |^n  \ar[u]^{|\msf{Eq}^n|} \ar[r]_{\sem{T}_0}	&\sets \ar[u]_{\msf{Eq}}  }
% \end{array}
% \]
% and
% \[
% \xymatrix@C+2pc{
% |\rel|^n \rtwocell^{\sem{\Gamma}_1}_{\sem{T}_1}{\;\;\; \sem{t}_1} & \mathsf{\Rel}
% }\msf{G}-set
% \]
% over
% \[
% \xymatrix@C+2pc{
% \rtwocell^{\Gamma_0}_{T_0}{\;\;\; t_0} & \sets
% }
% \]
%
%
%
%
%
%
%
% Following our recent work, we consider a parametric model of
% Units of Measure to be a reflexive graph of $\lambda D$-models. This will
% mean every type is interpreted as a fibred functor. Identity Extension
% corresponds to the fibred functor preserving some notion of equality,
% while the fundamental theorem of logical relations means every term is
% interpreted as a fibred natural transformation.

% \subsection{Kennedy's Model, Fibrationally}
% Andrew assumes a $\msf{Nat}$-indexed collection $\Psi \; n \subseteq
% \msf{Ab} n \ra \msf{\Rel}(Q, Q)$ of functions which he uses as
% environments to interpret types of the form $\qnt(u)$ as
% relations over $Q$. This family $\Psi$ is expected to satisfy a
% closure condition: if $S:m \ra \msf{Ab}\;n$ and $\phi \in \Psi\;n$,
% then $\phi \cdot S^* \in \Psi \; m$
%
% {\bf Question: I'd like to see this as a $\lambda D$-model but I cant see
% it.}\label{sec:MonSet}
% Andrew then shows how such a family gives a relational interpretation
% for every type: If $T$ is a type in $n$ free unit variables, then
% $\sem{T}_1:\Psi \; n \ra \Rel(\sem{T}_0, \sem{T}_0)$ where $\sem{T}_0$ is
% the interpretation of the $T$ in the erasure semantics described
% above. One might even write, for such a type $T$
%
%
% Andrew then shows that taking
% \[
% \Psi \; n = \{ \psi_k | k : n \ra Q^{>0} \} \; \mbox{ where } \;
% (q,q') \in \psi_k (u) \mbox{ iff }  q = (k^* u) q'
% \]
% allows him to prove the fundamental theorem of logical relations which
% in the above format is the following
% \[
% \xymatrix@C+2pc{
% \Psi n \rtwocell^{\Gamma_1}_{T_1}{\;\;\; t_1} & \mathsf{Rel}
% }
% \]
% over
% \[
% \xymatrix@C+2pc{
% 1 \rtwocell^{\Gamma_0}_{T_0}{\;\;\; t_0} & \sets
% }
% \]
% I
% saw nothing like identity extension though - in this example, I guess
% the identity extension lemma would state that $\sem{T}_1
% (\psi_{\lambda n . 1}) = \msf{Eq}_{\sem{T}_0}$. There is a more refined
% choice of $\Psi$ chosen for a specific signature but I'll not go into
% it. And notice that this identity extension property looks like it
% would complete a reflexive graph of $\lambda D$-models
% between Kennedy's relational model and his unit-erasure model - if
% Kennedy's relational model could be cast as a $\lambda D$-model.



% \section{$G$-Set Parametricity'}
%
% We have seen a model of dimension types based upon $G$-sets in the previous
% section. We now consider it as the basis of a parametric model.
%
% \begin{theorem}
%   Let $T$ be a type with $n$ free unit variables. and $\phi:G \times X
%   \ra X$ a $G$-set. Then there is a $G$-set $T_1 \phi : G^n \times T_0
%   \ra T_0$
% \end{theorem}
%
% \begin{proof}
% Induction on the structure of $T$. The only question pertains to the
% treatment of $\forall u . T$ since I haven’t decoded the right Kan
% extension yet.
% \end{proof}
%
% Note that here each type has an interpretation as a $G$-set which
% differs from Kennedy's approach where each type has an interpretation
% mapping indexed families of relations to relations.  We could get a
% "functional" model - i.e. one where the types are interpreted as
% functions - by mapping each $G$-set to the interpretation of the type
% in the model defined by that $G$-set. This amounts to not hard-wiring
% in the interpretation of $\qnt$ as an Abelian universe but allowing it to
% be a parameter over which the model varies. This in turns amounts to
% thinking of $\msf{num}$ not so much as a base type such as
% $\msf{Bool}$ but rather as a type variable, albeit indexed over an
% Abelian group. This leads to
%
% \begin{defn}
%   Let $\phi$-be a $G$-set and $T$ be a type with $n$-free type
%   variables. The mapping $\phi \mapsto T_1 \phi$ defines a $G$-set
%   semantics with the following lifting property
% \begin{center}
% \hspace{0.1in}
% \xymatrix{G^n\! -\!\msf{set} \ar[d] \ar[r]^{T_1} &G\! -\!\msf{set}\ar[d]\\ \C \ar[r]_{T_0}	&C}
% \end{center}
% where $\C$ is the semantic category interpreting types.
% \end{defn}
%
% {\bf Question: What is identity extension here? Would it arise from
%   taking the $G$-set $\pi_2:G \times X \ra X$ as the Abelian
%   universe. If so, we again need a functional interpretation of types
%   which once more suggests not hard-wiring in the interpretation of $\qnt$
% }
%
%
%
% \subsection{Bob's Model as a Reflexive Graph}
%
% \section{Algebraically Indexed Types, Fibrationally}
%
% \section{Lagrangians}
%
% \section{Application: Verification and Programming}
%
% {\bf Key idea 1}: The reverse function $\msf{rev} :: \msf{List} \; a \ra
% \msf{List}\; a$ preserves permutations because it is not just a
% function, but a $G$-set morphism with domain and codomain the obvious
% action of the permutation group on lists of a given length.
%
% Thus we essentially have an instance of parametricity where
% $\msf{rev}$ has a second lifted semantics in a category $G$-set over
% $\sets$. Can we develop some form of type system for typing $G$-set
% morphisms? Can we relate it to container morphisms - e.g. any container
% morphism whose action on positions is an isomorphism will preserve
% permutations. Other actions?
%
% {\bf Key idea 2:} To preserve permutations is to define a function on
% cyclic lists. Thus we get a paradigm for programming with quotient
% containers much simpler than, say, my previous work with Tarmo et
% al. But can we do $\msf{member}$. Yes!!!!



\bibliography{fibUoM}
\end{document}





